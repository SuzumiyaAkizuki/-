\documentclass[UTF8]{ctexart}
\usepackage{float}
\usepackage{listings}
\usepackage{graphicx}
\usepackage{xeCJK}
\usepackage{xcolor}
\usepackage{ulem}
\usepackage{CJKfntef}
\usepackage{setspace}
\usepackage{amsmath}
\lstset{
   basicstyle          =   \ttfamily,          % 基本代码风格
   keywordstyle        =   \ttfamily\bfseries,          % 关键字风格
   commentstyle        =   \rmfamily\itshape,  % 注释的风格,斜体
   stringstyle         =   \ttfamily,  % 字符串风格
   flexiblecolumns,                % 别问为什么,加上这个
   numbers             =   left,   % 行号的位置在左边
   showspaces          =   false,  % 是否显示空格,显示了有点乱,所以不现实了
   numberstyle         =   \zihao{-5}	\ttfamily ,    % 行号的样式,小五号,tt等宽字体
   showstringspaces    =   false,
   captionpos          =   t,      % 这段代码的名字所呈现的位置,t指的是top上面
   frame               =   lrtb,   % 显示边框
   breaklines      =   true,   % 自动换行,建议不要写太长的行
   columns         =   fixed,  % 如果不加这一句,字间距就不固定,很丑,必须加
}

\title{离散数学第七次作业}
\author{贠启豪 193775168}
\date{\today}

\begin{document}
    \maketitle
    \begin{enumerate}
        \item 判断以下公式是不是永真式、永假式、可满足式,并说明理由。
        \begin{enumerate}
            \item $ \exists xP(x)\vee \exists xQ(x)\rightarrow \exists x(P(x)\vee Q(x)) $
            
            永真。\\ 因为$\exists x(A \vee B)\Leftrightarrow \exists x A \vee \exists x B$且$(P\Leftrightarrow Q) \rightarrow (P\rightarrow Q)$
            \item $ \exists xP(x)\wedge \exists xQ(x)\rightarrow \exists x(P(x)\wedge Q(x))$
            
            可满足。
            
            对于解释$I_1$:
            \[
                D_{I_1}=\{1\},P^{I_1}(x)=1,Q^{I_1}(x)=1
            \]
            $I(\exists xP(x)\wedge \exists xQ(x)\rightarrow \exists x(P(x)\wedge Q(x)))=(1\rightarrow 1)=1$.

            对于解释$I_2$:
            \[
                D_{I_2}=\{0,1\},P^{I_2}(x)=x,Q^{I_2}(x)=\neg x
            \]
            $I(\exists xP(x)\wedge \exists xQ(x)\rightarrow \exists x(P(x)\wedge Q(x)))=(1\rightarrow 0)=0$.
            \item $\forall x(P(x)\vee Q(x))\rightarrow \forall xP(x)\vee \forall xQ(x)$
            
            可满足。
            
            对上题所用的解释$I_1$,公式为真。对于上题所用的解释$I_2$,公式为假。

            \item $ \forall xP(x,x)\rightarrow \forall x\forall yP(x,y)$
            
            可满足。
            
            对于解释$I_3$:
            \[
                D_{I_3}=R,P^{I_3}(x,y)=1
            \]
            $I(\forall xP(x,x)\rightarrow \forall x\forall yP(x,y))=(1\rightarrow 1)=1$

            对于解释$I_4$:
            \[
                D_{I_4}=R,P^{I_4}(x,y)=
                \begin{cases}
                    1, &x=y\\
                    0, &x\neq y\\
                \end{cases}
            \]
            $I(\forall xP(x,x)\rightarrow \forall x\forall yP(x,y))=(1\rightarrow 0)=0$
            \item $ (\forall xP(x)\rightarrow \forall xQ(x))\rightarrow \forall x(P(x)\rightarrow Q(x))$
            
            可满足。

            对于(1)中解释$I_1,I((\forall xP(x)\rightarrow \forall xQ(x))\rightarrow \forall x(P(x)\rightarrow Q(x)))=((1\rightarrow 1)\rightarrow 1)=1$.

            对于(1)中解释$I_2,I((\forall xP(x)\rightarrow \forall xQ(x))\rightarrow \forall x(P(x)\rightarrow Q(x)))=(0\rightarrow 0)\rightarrow 0=(1\rightarrow 0)=0$


            \item $ (\exists xP(x)\rightarrow \forall xQ(x))\rightarrow \forall x(P(x)\rightarrow Q(x))$
            
            永真。

            假设存在解释$I_5$使得$I(\forall x(P(x)\rightarrow Q(X)))=0$,则$\exists d\in D_{I_5},P(d)=1\wedge Q(d)=0$
            则$I(\exists x P(x))=1$且$\forall xQ(x)=0,I(\exists x P(x)\rightarrow \forall xQ(x))=0$.此时原公式为真。

            \item $ \forall x(P(x)\rightarrow Q(x))\rightarrow (\exists xP(x)\rightarrow \exists xQ(x))$
            
            永真。

            假设存在解释$I_6$,$I(\exists xP(x)\rightarrow \exists xQ(x))=0$,
            
            则$I(\exists xP(x))=1$且$I(\exists xQ(x))=0$.
            
            则$I(\forall x(P(x)\rightarrow Q(x)))=0$
            
            则原公式永真。
        \end{enumerate}

        \item 设A1是公式A的闭包,A2是$\exists x_1 \cdots \exists x_nA$,其中$Var(A)=\{x_1,…,x_n\}$。证明:
        \begin{enumerate}
            \item A是永真式当且仅当A1是永真式;
            
            证明:

            ($\rightarrow$) 设A是永真的,$x_i$是A中的一个自由变元。那么$\forall I,\forall \text{$I$中赋值}v,\forall d\in D_I$
            有$I(A)(v[x_i/d])=1$即$I(\forall x_i A)=1$。由$x_i$的任意性,有$\forall x_1 \cdots \forall x_n A$是永真的。
            
            
            ($\leftarrow$) 设$\forall x_1 \cdots \forall x_n A$永真,又因为$\forall x_1 \cdots \forall x_n A\rightarrow A$
            ,所以$A$永真。

            \item A是可满足式当且仅当A2是可满足式。
            证明:

            ($\rightarrow$) 设解释$I$和赋值$v$满足A,由于$A\rightarrow \exists x_1\cdots \exists x_n A$
            所以解释$I$和赋值$v$满足$A2$.

            ($\leftarrow$) 设解释$I$满足$A2$,有$d_1,\cdots d_n\in D_I,I(A)(\forall i,x_i/d_i)=1$。
            取赋值$v[x_i/d_i]$即可满足A。
        \end{enumerate}
        
        \item 下面谓词公式是否是永真式?是则证明之,不是,请举出反例:
        \begin{enumerate}
            \item $\exists x\forall yA(x,y)\leftrightarrow \forall y\exists xA(x,y)$
            
            不是。
            取解释$I_5$:
            \[
                D_{I_5}=R,A^{I_5}=\begin{cases}
                    1 & y=x-1\\
                    0 & \text{others}\\
                \end{cases}
            \]
            有$I(\forall y \exists xA(x,y))=1$且$\exists x \forall yA(x,y)=0$.

            \item $(\exists xA(x)\rightarrow \exists xB(x))\rightarrow \exists x(A(x)\rightarrow B(x))$
        
            不是。取解释$I_6$:
            \[
                D_{I_6}=R,A(x)=(x<0),B(x)=(x>0)
            \]
            则$I(\exists xA(x)\rightarrow \exists xB(x))=(1 \rightarrow 1)=1$,且
            $I(\exists x(A(x)\rightarrow B(x))=0$.
        \end{enumerate}

        \item 证明以下是永真式
        \begin{enumerate}
            \item $A_t^x\rightarrow \exists x A$
            
            证明:

            取解释$I$和赋值$v$,若有$I(A_t^x)(v)=1$,则$I(A)(v[x/I(t)(v)])=1$,则$\exists xA$,故$A_t^x\rightarrow \exists xA$

            \item $\neg \forall A \leftrightarrow \exists x \neg A$
            
            证明:
            对于解释$I$,赋值$v$
            \[
                \begin{aligned}
                    &I(\neg \forall A)(v)=1\\
                    &\Leftrightarrow I(\forall A)(v)=0\\
                    &\Leftrightarrow \exists d_i\in D_I,I(A)(v[x/d_i])=0\\
                    &\Leftrightarrow \exists d_i\in D_I,I(\neg A)(v[x/d_i])=0\\
                    &\Leftrightarrow I(\exists x \neg A)(v)=1\\
                \end{aligned}
            \]

            \item $\neg \exists xA\leftrightarrow \forall x\neg A$
            
            证明:对于解释$I$,赋值$v$,有:
            \[
                \begin{aligned}
                    &I(\neg \exists x A)(v)=1\\
                    &\Leftrightarrow I(\exists x A)(v)=0\\
                    &\Leftrightarrow \forall d_i\in D_I,I(A)(v[x/d_i])=0\\
                    &\Leftrightarrow \forall d_i\in D_I,I(\neg A)(v[x/d_i])=1\\
                    &\Leftrightarrow I(\forall x\neg A)(v)=1\\
                \end{aligned}
            \]
            
            \item $\exists x(A\wedge B)\rightarrow \exists xA\wedge \exists x B$
            
            证明:对于解释$I$,赋值$v$,有:
            \[
                \begin{aligned}
                    &I(\exists x(A\wedge B))(v)=1\\
                    &\Rightarrow \exists d_i\in D_I,I(A)(v[x/d_i])=I(B)(v[x/d_i])=1\\
                    &\Rightarrow I(\exists xA)(v)=I(\exists xB)(v)=1\\
                    &\Rightarrow I(\exists xA\wedge \exists xB)(v)=1
                \end{aligned}
            \]

            \item $\forall xA \vee \forall xB\rightarrow \forall x(A\vee B)$
            
            假设不是永真式,则存在解释$I$,赋值$v$,$I(\forall x(A\vee B))(v)=0$且$I(\forall xA\vee \forall xB)(v)=1$.
            \[
                \begin{aligned}
                    &I(\forall x(A\vee B))(v)=0\\
                    &\Rightarrow \exists d\in D_I,I(A\vee B)(v[x/d])=0\\
                    &\Rightarrow \exists d\in D_I,I(A)(v[x/d])=I(B)(v[x/d])=0\\
                    &\Rightarrow I(\forall xA\vee \forall xB)(v)=0
                \end{aligned}
            \]
            矛盾,假设错误。
            
            \item $\forall x(A\rightarrow B)\rightarrow (A\rightarrow \forall xB)$,其中x不是A的自由变元。
            
            假设不是永真式,则存在解释I和赋值v,使得$I(A\rightarrow \forall xB)(v)=0$且$I(\forall x(A\rightarrow B))(v)=1$.
            
            即$I(A)(v)=1$且$I(\forall xB)(v)=0$且$I(\forall x(A\rightarrow B))(v)=1$。
            
            假设对于解释I和赋值v,$I(\forall x(A\rightarrow B))(v)=I(A)(v)=1$,则$\forall d\in D_I,I(A\rightarrow B)(v[x/d])=1$
            由于$I(A)(v)=I(A)(v[x/d])=1$,所以$I(B)(v[x/d])=1$,矛盾,假设错误。
            
        \end{enumerate}
        
        \item 下面公式是否是永真式?说明理由。
        \begin{enumerate}
            \item $(A\rightarrow \exists xB(x))\leftrightarrow \exists x(A\rightarrow B(x))$
            
            是永真的。
            \[
                \begin{aligned}
                    &\exists x(A\rightarrow B(x))\\
                    &\Leftrightarrow \exists x(\neg B(x)\rightarrow \neg A)\\
                    &\Leftrightarrow \forall x\neg B(x)\rightarrow \neg A\\
                    &\Leftrightarrow \neg \neg A\rightarrow \neg \forall x \neg B\\
                    &\Leftrightarrow A\rightarrow \exists xB(x)\\
                \end{aligned}
            \]
            
            \item $\exists x(A(x)\rightarrow B(x))\leftrightarrow (\forall xA(x)\rightarrow \exists xB(x)$)
            
            永真。
            \[
                \begin{aligned}
                    &I(\exists x(A\rightarrow B))(v)=0\\
                    &\Leftrightarrow \forall d\in D_I,I(A(x))(v[x/d])=1 \text{ and } I(B)(v[x/d])=0\\
                    &\Leftrightarrow I(\forall xA)(v)=1 \text{ and } I(\exists xB)(v)=0\\
                    &\Leftrightarrow I(\forall xA\rightarrow \exists xB)(v)=0
                \end{aligned}
            \]
            
            \item $\forall x(A(x)\wedge B(x))\leftrightarrow \forall xA(x)\wedge \forall xB(x)$
            
            永真。
            \[
                \begin{aligned}
                    &I(\forall x(A(x)\wedge B(x)))=0\\
                    &\Leftrightarrow \exists d\in D_I,I(A\wedge B)(v[x/d])=0\\
                    &\Leftrightarrow \exists d\in D_I,I(A)(v[x/d])=0 \text{ or } I(B)(v[x/d])=0\\
                    &\Leftrightarrow I(\forall xA)(v)=0 \text{or} I(\forall xB)(v)=0\\
                    &\Leftrightarrow I(\forall xA\wedge \forall xB)(v)=0
                \end{aligned}
            \]
        \end{enumerate}
        
        
    \end{enumerate}


\end{document}