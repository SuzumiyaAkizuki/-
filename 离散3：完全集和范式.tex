\documentclass[UTF8]{ctexart}
\usepackage{float}
\usepackage{listings}
\usepackage{graphicx}
\usepackage{xeCJK}
\usepackage{xcolor}
\usepackage{ulem}
\usepackage{CJKfntef}
\usepackage{setspace}
\usepackage{amsmath}
\lstset{
   basicstyle          =   \ttfamily,          % 基本代码风格
   keywordstyle        =   \ttfamily\bfseries,          % 关键字风格
   commentstyle        =   \rmfamily\itshape,  % 注释的风格,斜体
   stringstyle         =   \ttfamily,  % 字符串风格
   flexiblecolumns,                % 别问为什么,加上这个
   numbers             =   left,   % 行号的位置在左边
   showspaces          =   false,  % 是否显示空格,显示了有点乱,所以不现实了
   numberstyle         =   \zihao{-5}	tfamily,    % 行号的样式,小五号,tt等宽字体
   showstringspaces    =   false,
   captionpos          =   t,      % 这段代码的名字所呈现的位置,t指的是top上面
   frame               =   lrtb,   % 显示边框
   breaklines      =   true,   % 自动换行,建议不要写太长的行
   columns         =   fixed,  % 如果不加这一句,字间距就不固定,很丑,必须加
}

\title{离散数学第三次作业}
\author{贠启豪 19375168}
\date{\today}

\begin{document}
   \maketitle
   \begin{enumerate}
        \item 证明以下联结词集合是极小完全集。
        \begin{enumerate}
           \item $\{0 , \rightarrow\}$
           
           证明:由于$\neg p\Leftrightarrow p\rightarrow 0$,且$\{\neg , \rightarrow\}$
           是完全集,所以$\{0, \rightarrow \}$是完全集。

           如果任取由$\{0\}$生成的公式$A$,则$A\Leftrightarrow 0$,对
           于真值赋值$v(p/0),v(\neg p)=1,v(A)=0$,所以$\{0\}$不是完全集。

           如果任取只由$p$和$\{\rightarrow \}$生成的公式A,对于
           真值赋值$v(p/1),v(A)=1,v(\neg p)=0$,所以$\{\rightarrow \}$
           不是完全集。

           所以$\{0 , \rightarrow\}$是极小完全集。

           \item $\{\oplus , \rightarrow\}$
           
            证明:由于$\neg p\Leftrightarrow p\oplus 1\Leftrightarrow p\oplus(p\rightarrow p)$,且$\neg,\rightarrow$是完全集,所以$\{\oplus,\rightarrow\}$是完全集。

            任取由$\{\oplus \}$和$p$生成,而且p出现了偶数次的公式$A$,对于真值赋值
            $v(p/1),有v(A)=1,v(\neg p)=0$,所以$\{\oplus \}$不是完全集。

            如果任取只由$p$和$\{\rightarrow \}$生成的公式A,对于
           真值赋值$v(p/1),v(A)=1,v(\neg p)=0$,所以$\{\rightarrow \}$
           不是完全集。

           所以$\{\oplus ,\rightarrow\}$是极小完全集。

           \item $\{\oplus , \wedge ,  \leftrightarrow \}$
           
           证明:由于$\neg p\Leftrightarrow p\oplus 1\Leftrightarrow p\oplus (p\leftrightarrow p)$,
           而且$\{\neg,\wedge \}$是完全集,所以$\{\oplus , \wedge ,  \leftrightarrow \}$
           是完全集。

           任取由$\{\oplus \wedge\}$和命题变元$p$生成的公式A,对于真值赋值
           $v(p/0),v(A)=0,v(\neg p)=1$,所以$\{\oplus, \wedge\}$不是完全集。

           任取由$\{\wedge \oplus\}$和命题变元$p$生成的公式$A$,对于真值赋值
           $v(p/1),v(A)=1,v(\neg p)=0$,所以$\{\wedge, \oplus\}$不是完全集。

           $\{\oplus \leftrightarrow\}$不是完全集。

           所以$\{\oplus,\wedge,\leftrightarrow\}$是极小完全集。
           
           \item $\{\oplus, \vee ,  \leftrightarrow \}$
           
           证明:由于$\neg p\Leftrightarrow p\oplus 1\Leftrightarrow p\oplus (p\leftrightarrow p)$,
           而且$\{\neg,\vee \}$是完全集,所以$\{\oplus , \vee ,  \leftrightarrow \}$
           是完全集。

           对于任意由$\{\oplus,\vee\}$和命题变元$p$生成的公式$A$,对于真值赋值
           $v(p/0),v(A)=0,v(\neg p)=1$,所以$\{\oplus,\vee\}$不是完全集。

           对于任意由$\{\vee,\leftrightarrow\}$和命题变元$p$生成的公式A,对于
           真值赋值$v(p/1),v(A)=1,v(\neg p)=0$,所以$\{\vee,\leftrightarrow\}$
           不是完全集。

        
        
        \end{enumerate}
        \item  证明以下联结词集合不是完全集。
        \begin{enumerate}
           \item $\{\wedge, \vee, \rightarrow, \leftrightarrow\}$
           
           对于任意由上述集合和命题变元$p$形成的公式$A$,
           对于真值赋值$v(p/1),有v(A)=1,v(\neg p)=0$,所以上述集合不是
           完全集。
           
           \item $ \{\oplus , \wedge, \vee\}$
           
           对于任意由上述集合和命题变元$p$形成的公式$A$,
           对于真值赋值$v(p/0),有v(A)=0,v(\neg p)=1$,所以上述集合不是
           完全集。
        \end{enumerate}    
        \item 二元联结词­­↑(称为“与非”)和↓(称为“或非”)的真值表如下。
        
        \begin{table}[H]
            \centering
            \begin{tabular}{|cccc|}
                \hline
                $p$ & $q$ & $p \uparrow q$ & $p\downarrow q$\\
                \hline
                0 & 0 & 1 & 1\\
                \hline
                0 & 1 & 1 & 0\\
                \hline
                1 & 0 & 1 & 0\\
                \hline
                1 & 1 & 0 & 0\\
                \hline
            \end{tabular}
        \end{table}
        证明:
        \begin{enumerate}
            \item $\{\uparrow \}$是完全集
            
            证明:由于$\neg p\Leftrightarrow (p \uparrow p),p\wedge q\Leftrightarrow \neg(p\uparrow q)\Leftrightarrow (p\uparrow q)\uparrow(p\uparrow q)$
            
            所以$\{\uparrow \}$是完全集。

            \item $\{\downarrow \}$是完全集
            
            证明:由于$\neg p\Leftrightarrow (p \downarrow p),p\vee q\Leftrightarrow \neg(p\downarrow q)\Leftrightarrow (p\downarrow q)\downarrow(p\downarrow q)$
            
            所以$\{\uparrow \}$是完全集。
            
            \item 如果二元连接词$\{\Delta\}$是完全集,那么它只能是$\uparrow$或者$\downarrow$.
            
            证明:若$0\Delta 0\Leftrightarrow 0$或者$1\Delta 1\Leftrightarrow 1$
            那么$\neg$不能由$\Delta$定义。所以一定有$0\Delta 0 = 1$且$1\Delta 1=0$

            如果$1\Delta 0$不等于$0 \Delta 1$,则其真值表最后有2个1,真值表中有1个1的$\wedge$无法定义。
            所以一定有$1\Delta 0 \Leftrightarrow 0\Delta 1$

            若$1\Delta 0=1$,则$\Delta$是$\uparrow$

            若$1\Delta 0=0$,则$\Delta$是$\downarrow$
            
        \end{enumerate}
      \item 证明三元连接词$\Delta$是极小完全集,其真值表如下:
      
      \begin{table}[H]
         \centering
         \begin{tabular}{|cccc|}
            \hline
            $p$ & $q$ & $r$ & $\Delta(p,q,r)$ \\
            \hline
            0 & 0 & 0 & 1\\
            \hline
            0 & 0 & 0 & 1\\
            \hline
            0 &1& 0& 0\\
            \hline
            0 & 1 & 1 & 0\\ 
            \hline
            1 & 0 & 0 & 0\\
            \hline
            1 & 0 & 1 & 0\\
            \hline
            1 & 1 & 0 & 1\\
            \hline
            1 & 1 & 1 & 0\\
            \hline
         \end{tabular}
      \end{table}
      
      证明:经过观察,$r\uparrow p\Leftrightarrow \Delta(p,p,r)$,而$\{\uparrow \}$是完全集。
      所以$\{\Delta \}$是极小完全集。

      \item 在下列公式中,哪些是析取范式,哪些是合取范式?
      \begin{enumerate}
         \item $p$ 析取范式
         \item $p\vee q$ 析取范式
         \item $(p\vee q)\wedge r$ 合取范式
         \item $p\wedge \neg r$ 合取范式
         \item $p \vee \neg p$ 析取范式
         \item $(( p \vee q) \wedge \neg q) \vee r$ 不是范式
      \end{enumerate}
      
      \item 在下列公式中,哪些是关于 $p, q, r$ 的主析取范式,哪些是关于 $p, q, r$ 的主合取范式?
      \begin{enumerate}
         \item $p \vee q \vee r$ 主合取范式
         \item $p \wedge \neg q \wedge r$ 主析取范式
         \item $( p \vee q \vee \neg r ) \wedge ( p \vee q \vee \neg r)$ 主合取范式
      \end{enumerate}
      其他的都不是$p,q,r$的主范式。
      
      \item 是否有这样的公式,它既是主合取范式,又是主析取范式?如果有,举出一例。

      解:有,比如只含有一个命题变元的公式:$p$。

      \item 求下列公式的主范式,进而判断其是否永真式、永假式、可满足式。
      \begin{enumerate}
         \item $ \neg p \wedge   q \rightarrow  r$
         \[
            \begin{aligned}
               &\mathrel{\phantom{=}}\neg p \wedge   q \rightarrow  r\\
               &\Leftrightarrow \neg(\neg p \wedge q)\vee r\\
               &\Leftrightarrow (p\vee \neg q \vee r)\text{(主合取范式)}\\
            \end{aligned}
         \]
         是可满足式。
         \item $ ( p \rightarrow  q) \rightarrow  r$
         \[
            \begin{aligned}
               &\mathrel{\phantom{=}}( p \rightarrow  q) \rightarrow  r\\
               &\Leftrightarrow \neg(\neg p\vee q)\vee r\\
               &\Leftrightarrow (p \wedge \neg q)\vee r\\
               &\Leftrightarrow (p \vee r)\wedge (\neg q \vee r)\\
               &\Leftrightarrow (p \vee r \vee (q\wedge \neg q)) \wedge (\neg q \vee r \vee (p \wedge \neg p))\\
               &\Leftrightarrow (p \vee r \vee q) \wedge (p\vee r \vee \neg q) \wedge (\neg q \vee r \vee p)\wedge(\neg q \vee r \vee \neg p)\text{主合取范式}\\
            \end{aligned}
         \]
         主合取包含三个极小式,是可满足式。
         \item $\neg p \vee  \neg q \rightarrow  ( p \leftrightarrow  \neg q)$
         \[
            \begin{aligned}
               &\mathrel{\phantom{=}}\neg p \vee  \neg q \rightarrow  ( p \leftrightarrow  \neg q)\\
               &\Leftrightarrow \neg (\neg p \vee \neg q)\vee (\neg p \vee \neg q) \wedge (q \vee p)\\
               &\Leftrightarrow (p \wedge q)\vee (p \vee q)\\
               &\Leftrightarrow p\vee q\ \text{主合取范式}
            \end{aligned}
         \]
         是可满足式。

         \item $p \vee  ( p \rightarrow  q \vee  (\neg q \rightarrow  r))$
         \[
            \begin{aligned}
               &\mathrel{\phantom{=}}p \vee  ( p \rightarrow  q \vee  (\neg q \rightarrow  r))\\
               &\Leftrightarrow p\vee (\neg p \vee q \vee (q\vee r))\\
               &\Leftrightarrow p\vee \neg p \vee q \vee r\\
               &\Leftrightarrow (p\vee (q \wedge \neg q) \vee (r\wedge \neg r))\vee (\neg p \vee q \vee r)\\
               &\Leftrightarrow(p\vee q \vee r)\wedge(p\vee q \vee \neg r)\wedge(p\vee \neg q \vee r)\wedge(p\vee \neg q\vee \neg r)\wedge(\neg p \vee q \vee r)\\
            \end{aligned}
         \]
         是永真式。

         \item $( p \rightarrow  q \wedge   r) \wedge   (\neg p \rightarrow  \neg q \wedge   \neg r)$
         \[
           \begin{aligned}
               &\mathrel{\phantom{=}}( p \rightarrow  q \wedge   r) \wedge   (\neg p \rightarrow  \neg q \wedge   \neg r)\\
               &\Leftrightarrow(\neg p \vee (q\wedge r))\wedge(p\vee (\neg q \wedge \neg r))\\
               &\Leftrightarrow (\neg p\vee q)\wedge(\neg p \vee r)\wedge (p\vee \neg q) \wedge (p\vee \neg r)\\
               &\Leftrightarrow(p \wedge q \wedge r)\vee (\neg p \wedge \neg q \wedge \neg r)\ \text{主析取范式}\\
           \end{aligned} 
         \]
         是可满足式。
         \item $p \wedge   q \wedge   (\neg p \vee  \neg q)$
         \[
           \begin{aligned}
               &\mathrel{\phantom{=}}p \wedge   q \wedge   (\neg p \vee  \neg q)\\
               &\Leftrightarrow (p\vee (q \wedge \neg q))\wedge (q \vee (p \wedge \neg p))\wedge (\neg p \vee \neg q)\\
               &\Leftrightarrow (p \vee q) \wedge (p \vee \neg q) \wedge (\neg p \vee q) \wedge (\neg p \vee \neg q)\ \text{主合取范式}\\
           \end{aligned} 
         \]
         是永假式。
      \end{enumerate}
      \item 用主范式证明下列等值式。
      \begin{enumerate}
         \item $( p \rightarrow  q) \rightarrow  p \wedge  q \Leftrightarrow  (\neg p \rightarrow  p) \wedge  (r \rightarrow  p)$
         \[
            \begin{aligned}
               &\mathrel{\phantom{=}}( p \rightarrow  q) \rightarrow  p\\
               &\Leftrightarrow \neg(\neg p \vee q)\vee (p \wedge q)\\
               &\Leftrightarrow (p\wedge \neg q)\vee(p\wedge q)\\
               &\Leftrightarrow (p \wedge \neg q \wedge (r \vee \neg r))\vee (p \wedge q \wedge (r\vee \neg r))\\
               &\Leftrightarrow (p \wedge \neg q \wedge r)\vee(p\wedge \neg q \wedge \neg r)\vee(p \wedge q \wedge r)\vee(p\wedge q\wedge \neg r)\\
            \end{aligned}
         \]
         与此同时,有:
         \[
            \begin{aligned}
               &\mathrel{\phantom{=}}(\neg p \rightarrow  p) \wedge  (r \rightarrow  p)\\
               &\Leftrightarrow (p\vee p)\wedge (\neg r \vee p)\\
               &\Leftrightarrow p\wedge (p\vee \neg r)\\
               &\Leftrightarrow p\\
               &\Leftrightarrow p\vee(q \wedge \neg q)\vee(r\wedge \neg r)\\
               &\Leftrightarrow (p \wedge \neg q \wedge r)\vee(p\wedge \neg q \wedge \neg r)\vee(p \wedge q \wedge r)\vee(p\wedge q\wedge \neg r)\\
            \end{aligned}
         \]
         他们都等价于一个主析取范式$(p \wedge \neg q \wedge r)\vee(p\wedge \neg q \wedge \neg r)\vee(p \wedge q \wedge r)\vee(p\wedge q\wedge \neg r)$,证毕。
         \item $( p \rightarrow  q) \wedge  ( p \rightarrow  r) \Leftrightarrow  p \rightarrow  q \wedge  r$
         \[
            \begin{aligned}
               &\mathrel{\phantom{=}}( p \rightarrow  q) \wedge  ( p \rightarrow  r)\\
               &\Leftrightarrow (\neg p \vee q)\wedge (\neg p \vee r)\\
            \end{aligned}
         \]
         与此同时,有
         \[
            \begin{aligned}
               &\mathrel{\phantom{=}}p \rightarrow  q \wedge  r\\
               &\Leftrightarrow \neg p \vee(q\wedge r)\\
               &\Leftrightarrow (\neg p \vee q)\wedge (\neg p \vee r)\\
            \end{aligned}
         \]
         他们都等价于一个主析取范式$(\neg p \vee q)\wedge (\neg p \vee r)$,证毕。
      \end{enumerate}
      
      
      
     
   \end{enumerate}
   

\end{document}