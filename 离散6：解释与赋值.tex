\documentclass[UTF8]{ctexart}
\usepackage{float}
\usepackage{listings}
\usepackage{graphicx}
\usepackage{xeCJK}
\usepackage{xcolor}
\usepackage{ulem}
\usepackage{CJKfntef}
\usepackage{setspace}
\usepackage{amsmath}
\lstset{
   basicstyle          =   \ttfamily,          % 基本代码风格
   keywordstyle        =   \ttfamily\bfseries,          % 关键字风格
   commentstyle        =   \rmfamily\itshape,  % 注释的风格,斜体
   stringstyle         =   \ttfamily,  % 字符串风格
   flexiblecolumns,                % 别问为什么,加上这个
   numbers             =   left,   % 行号的位置在左边
   showspaces          =   false,  % 是否显示空格,显示了有点乱,所以不现实了
   numberstyle         =   \zihao{-5}	tfamily,    % 行号的样式,小五号,tt等宽字体
   showstringspaces    =   false,
   captionpos          =   t,      % 这段代码的名字所呈现的位置,t指的是top上面
   frame               =   lrtb,   % 显示边框
   breaklines      =   true,   % 自动换行,建议不要写太长的行
   columns         =   fixed,  % 如果不加这一句,字间距就不固定,很丑,必须加
}

\title{离散数学第六次作业}
\author{凉宫秋月}
\date{\today}

\begin{document}
    \maketitle
    \begin{enumerate}
        \item 给定解释I和I中赋值v如下:
        \[
            D_I=\{ 1,2\},a^I=1,b^I=2,f^I(1)=2,f^I(2)=1
        \]
        \[
            P^I(1,1)=P^I(1,2)=1,P^I(2,1)=P^I(2,2)=0,v(x)=1,v(y)=1
        \]
        计算下列公式在解释I,赋值v下的真值。
        \begin{enumerate}
            \item $P(a, f (x)) \wedge P(x, f (b)) \wedge P( f ( y), x)$
            \[
                \begin{aligned}
                    &P(a, f (x)) \wedge P(x, f (b)) \wedge P( f ( y), x)\\
                    &=P(1,2)\wedge P(1,1) \wedge P(2,1)\\
                    &=1\wedge 1\wedge 0\\
                    &=0
                \end{aligned}
            \]
            \item $\forall x \exists yP( y, x)$
            \[
                \begin{aligned}
                    &I(\forall x \exists yP( y, x))(v)\\
                    &=I(\exists yP(y,x))v[x/1] \wedge I(\exists yP(y,x))v[x/2]\\
                    &=(I(P(y,x))v[y/1,x/1] \vee I(P(y,x))v[y/2,x/1])\\
                    &\wedge (I(P(y,x))v[y/1,x/2] \vee I(P(y,x))v[y/2,x/2])\\
                    &=(1\vee 0)\wedge (1\vee 0)\\
                    &=1
                \end{aligned}
            \]
            \item $\forall x\forall y(P(x, y) \rightarrow P( f (x), f (y)))$
            \[
                \begin{aligned}
                    &I(\forall x\forall y(P(x, y) \rightarrow P( f (x), f (y))))(v)\\
                    &=I(P(x,y)\rightarrow P(f(x),f(y)))v[x/1,y/1]\wedge I(P(x,y)\rightarrow P(f(x),f(y)))v[x/1,y/2]\\
                    &\wedge I(P(x,y)\rightarrow P(f(x),f(y)))v[x/2,y/1] \wedge I(P(x,y)\rightarrow P(f(x),f(y)))v[x/2,y/2]\\
                    &=0\wedge \cdots\\
                    &=0 
                \end{aligned}
            \]
        \end{enumerate}
        
        \item 给定解释I如下:
        \[
            D_I={a,b},P^I(a,a)=P^I(b,b)=1,P^I(a,b)=P^I(b,a)=0
        \]
        判断I是不是以下语句的模型。
        \begin{enumerate}
            \item $\forall x \exists yP(x,y)$  是
            \item $\forall x \forall y P(x,y)$ 否
            \item $\exists x \forall y P(x,y)$ 否
            \item $\exists x \exists y \neg P(x,y)$ 是
            \item $\forall x \forall y (P(x,y)\rightarrow P(y,x))$ 是
            \item $\forall x P(x,x)$ 是
        \end{enumerate}
        \item 写出一个语句 A,使得 A 有模型,并且 A 的每个模型的论域至少有三个元素。
        \[
            A=\neg F(x,x) \wedge F(a,b) \wedge F(b,c) \wedge F(a,c)
        \]
        其中一个A的模型为:
        \[
            D_I=N,F(a,b)=(a\neq b),a^I=1,b^I=2,c^I=3
        \]
        只要解释中a,b,c是三个互不相同的数,那么A总成立,此时解释I的论域中至少包含a,b,c三个数。

        \item 写出一个语句 A,使得 A 有模型,并且 A 的每个模型的论域有无穷多个元素。
        \[
            A=(\forall x \neg P(x,x)) \wedge (\forall x \forall y (P(x,y)\wedge P(y,z)\rightarrow P(x,z))) \wedge \forall x \exists y P(x,y)
        \]
        其中一个A的模型为:
        \[
            D_I=R,P(a,b)=(a<b)
        \]
        对于模型I,$\forall x_1\in D_I,\exists x_2\in D_I,P(x_1,x_2)=1$,对上述$x_2,\exists x_3,P(x_2,x_3)=1$,且由于$(\forall x \forall y (P(x,y)\wedge P(y,z)\rightarrow P(x,z)))$
        ,$P(x_1,x_3)=1$,则有$x_1<x_2<x_3$,此过程可以不断进行。
    \end{enumerate}


\end{document}