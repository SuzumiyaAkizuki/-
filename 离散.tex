\documentclass[UTF8]{ctexart}
\usepackage{float}
\usepackage{listings}
\usepackage{graphicx}
\usepackage{xeCJK}
\usepackage{xcolor}
\usepackage{ulem}
\usepackage{CJKfntef}
\usepackage{setspace}
\usepackage{amsmath}
\lstset{
   basicstyle          =   \ttfamily,          % 基本代码风格
   keywordstyle        =   \ttfamily\bfseries,          % 关键字风格
   commentstyle        =   \rmfamily\itshape,  % 注释的风格,斜体
   stringstyle         =   \ttfamily,  % 字符串风格
   flexiblecolumns,                % 别问为什么,加上这个
   numbers             =   left,   % 行号的位置在左边
   showspaces          =   false,  % 是否显示空格,显示了有点乱,所以不现实了
   numberstyle         =   \zihao{-5}	tfamily,    % 行号的样式,小五号,tt等宽字体
   showstringspaces    =   false,
   captionpos          =   t,      % 这段代码的名字所呈现的位置,t指的是top上面
   frame               =   lrtb,   % 显示边框
   breaklines      =   true,   % 自动换行,建议不要写太长的行
   columns         =   fixed,  % 如果不加这一句,字间距就不固定,很丑,必须加
}

\title{离散数学第二次作业}
\author{贠启豪 19375168}
\date{\today}

\begin{document}
   \maketitle
    \begin{enumerate}
        \item 用真值表证明下列公式
        \begin{enumerate}
            \item $p\wedge (q\oplus r)\Leftrightarrow (p \wedge q)\oplus (p\wedge r) $
            
            解:
            \begin{table}[H]
                \centering
                \begin{tabular}{|cccccccc|}
                    \hline
                    $p$ & $q$ & $r$ & $q\oplus r$ & $p\wedge (q\oplus r)$ & $p\wedge q$ & $p\wedge r$ & $(p \wedge q)\oplus (p\wedge r)$ \\
                    \hline
                    0 & 0 & 0 & 0 & 0 & 0 & 0 & 0 \\ 
                    \hline
                    0 & 0 & 1 & 1 & 0 & 0 & 0 & 0 \\
                    \hline
                    0 & 1 & 0 & 1 & 0 & 0 & 0 & 0 \\
                    \hline
                    0 & 1 & 1 & 0 & 0 & 0 & 0 & 0 \\
                    \hline
                    1 & 0 & 0 & 0 & 0 & 0 & 0 & 0 \\
                    \hline
                    1 & 0 & 1 & 1 & 1 & 0 & 1 & 1 \\
                    \hline
                    1 & 1 & 0 & 1 & 1 & 1 & 0 & 1 \\
                    \hline
                    1 & 1 & 1 & 0 & 0 & 1 & 1 & 0 \\
                    \hline
                \end{tabular}
            \end{table}
            真值表如上图所示,可以看出,$p\wedge (q\oplus r)$
            的取值总和$(p \wedge q)\oplus (p\wedge r)$
            相同,故有
            \[
                p\wedge (q\oplus r)\Leftrightarrow (p \wedge q)\oplus (p\wedge r)
            \]

            \item $p\oplus 1 \Leftrightarrow \neg p$
            
            解:
            \begin{table}[H]
                \centering
                \begin{tabular}{|ccc|}
                    \hline
                    $p$ & $p\oplus 1$ & $\neg p$ \\
                    \hline
                    0 & 1 & 1 \\ 
                    \hline
                    1 & 0 & 0 \\
                    \hline
                \end{tabular}
            \end{table}
            真值表如上图所示,可以看出,$p\oplus 1$
            的取值总和$\neg p$
            相同,故有
            \[
                p\oplus 1 \Leftrightarrow \neg p
            \]

            \item $p \vee (p\wedge q)\Leftrightarrow p$
            
            解:

            \begin{table}[H]
                \centering
                \begin{tabular}{|ccccc|}
                    \hline
                    $p$ & $q$ & $p\wedge q$ & $p\vee (p\wedge q)$ & $p$ \\
                    \hline
                    0 & 0 & 0 &0 & 0 \\ 
                    \hline
                    0 & 1 & 0 &0 & 0 \\
                    \hline
                    1 & 0 & 0 &1 & 1 \\
                    \hline
                    1 & 1 & 1 &1 & 1 \\
                    \hline
                \end{tabular}
            \end{table}
            真值表如上图所示,可以看出,$p\vee (p\wedge q)$
            的取值总和$p$
            相同,故有
            \[
                p \vee (p\wedge q)\Leftrightarrow p
            \]

            \item $p\oplus q \Leftrightarrow \neg(p\leftrightarrow q)$
            
            解:

            \begin{table}[H]
                \centering
                \begin{tabular}{|ccccc|}
                    \hline
                    $p$ & $q$ & $p\oplus q$ & $p \leftrightarrow q$ & $\neg (p\leftrightarrow q)$ \\
                    \hline
                    0 & 0 & 0 &1 & 0 \\ 
                    \hline
                    0 & 1 & 1 &0 & 1 \\
                    \hline
                    1 & 0 & 1 &0 & 1 \\
                    \hline
                    1 & 1 & 0 &1 & 0 \\
                    \hline
                \end{tabular}
            \end{table}
            真值表如上图所示,可以看出,$p\oplus q$
            的取值总和$\neg (p \leftrightarrow q)$
            相同,故有
            \[
                p\oplus q \Leftrightarrow \neg(p\leftrightarrow q)
            \]
        \end{enumerate}



    \item 用等值演算证明以下等值式
        \begin{enumerate}
            \item $p \rightarrow (q \rightarrow r) \Leftrightarrow q \rightarrow ( p \rightarrow r)$
            
            \[  
                \begin{aligned}
                &\mathrel{\phantom{=}}p\rightarrow q(p\rightarrow r)\\
                 &\Leftrightarrow \neg p \vee (\neg q \vee r)\\
                 &\Leftrightarrow (\neg p \vee \neg q)\vee r\\
                 &\Leftrightarrow \neg q \vee (\neg p \vee r)\\
                 &\Leftrightarrow q \rightarrow (p \rightarrow r)\\
                \end{aligned}
             \]

            \item $( p \rightarrow q) \wedge ( p \rightarrow r) \Leftrightarrow p \rightarrow q \wedge r$
            \[
                \begin{aligned}
                    &\mathrel{\phantom{=}}(p\rightarrow q)\wedge (q\rightarrow r)\\
                    &\Leftrightarrow (\neg p \vee q)\wedge (\neg p \vee r)\\
                    &\Leftrightarrow \neg p \vee (q \wedge r)\\
                    &\Leftrightarrow p\rightarrow q \wedge r
                \end{aligned}
            \]
            
            
            \item  $( p \rightarrow q) \vee (r \rightarrow q) \Leftrightarrow p \wedge r \rightarrow q$
            
            \[
                \begin{aligned}
                    &\mathrel{\phantom{=}} (p\rightarrow q)\wedge (r \rightarrow q)\\
                    &\Leftrightarrow (\neg p \vee q)\vee (\neg r \vee q)\\
                    &\Leftrightarrow \neg p \vee \neg r \vee q\\
                    &\Leftrightarrow \neg(p\wedge r)\vee q\\
                    &\Leftrightarrow p\wedge r \rightarrow q\\
                \end{aligned}
            \]
            
            \item  $p \rightarrow (q \rightarrow p) \Leftrightarrow \neg p \rightarrow ( p \rightarrow q)$
            
            \[
                \begin{aligned}
                    &\mathrel{\phantom{=}} p\rightarrow (q\rightarrow p)\\
                    &\Leftrightarrow \neg p \vee (\neg q \vee p)\\
                    &\Leftrightarrow 1\\
                    &\Leftrightarrow p \vee \neg p \vee q \\
                    &\Leftrightarrow \neg \neg p \vee (\neg p \vee q)\\
                    &\Leftrightarrow \neg p \rightarrow (p \rightarrow q)\\
                \end{aligned}
            \]
            
            \item  $( p \rightarrow q) \wedge (r \rightarrow q) \Leftrightarrow p \vee r \rightarrow q$
            \[
                \begin{aligned}
                    &\mathrel{\phantom{=}} (p\rightarrow q) \wedge (r\wedge q)\\
                    &\Leftrightarrow (\neg p \vee q) \wedge (\neg r \vee q)\\
                    &\Leftrightarrow (\neg p \wedge \neg r) \vee q\\
                    &\Leftrightarrow \neg(p \vee r)\vee q\\
                    &\Leftrightarrow (p\vee r)\rightarrow q\\
                \end{aligned}
            \]
            \item  $\neg( p \leftrightarrow q) \Leftrightarrow p \leftrightarrow \neg q$
            \[
                \begin{aligned}
                    &\mathrel{\phantom{=}}\neg (p\leftrightarrow q)\\
                    &\Leftrightarrow p \oplus q\\
                    &\Leftrightarrow (p \oplus q )\oplus (1 \oplus 1)\\
                    &\Leftrightarrow (p \oplus (q\oplus 1))\oplus 1\\
                    &\Leftrightarrow \neg (p \oplus \neg q)\\
                    &\Leftrightarrow \neg \neg (p \leftrightarrow \neg q)\\
                    &\Leftrightarrow p \leftrightarrow \neg q\\
                \end{aligned}    
            \]
        \end{enumerate}
    
    \item 用等值演算证明以下公式是永真式
    \begin{enumerate}
        \item $( q \rightarrow  p ) \wedge  ( \neg p \rightarrow  q ) \leftrightarrow p$
        \[    
            \begin{aligned}
                &\mathrel{\phantom{=}} (q\rightarrow p)\wedge (\neg p \rightarrow q) \leftrightarrow p\\
                &\Leftrightarrow (\neg q \vee p)\wedge (p \vee q) \leftrightarrow p\\
                &\Leftrightarrow (\neg q \wedge q)\vee p \leftrightarrow p\\
                &\Leftrightarrow 0\wedge p \leftrightarrow p\\
                &\Leftrightarrow p\wedge p\\
                &\Leftrightarrow 1\\
            \end{aligned}
        \]

        \item $( p \rightarrow  q) \wedge  (r \rightarrow  s) \rightarrow  ( p \wedge  r \rightarrow  q \wedge  s)$
        \[
            \begin{aligned}
                &\mathrel{\phantom{=}}(p\rightarrow q)\wedge (r \rightarrow s)\rightarrow(p\wedge r \rightarrow q\wedge s)\\
                &\Leftrightarrow \neg((\neg p \vee q)\wedge (\neg r\vee s))\vee(\neg(p\wedge r)\vee (q \wedge s))\\
                &\Leftrightarrow \neg(\neg p\vee q) \vee \neg (\neg r \vee s)\vee (\neg p \vee \neg r \vee (q \wedge s))\\
                &\Leftrightarrow (p\wedge \neg q) \vee (r \wedge \neg s) \vee \neg p \vee \neg r \vee (q \wedge s)\\
                &\Leftrightarrow \neg p \vee \neg q \vee \neg r \vee \neg s \vee (q\wedge s)\\
                &\Leftrightarrow \neg(q\wedge s) \vee (q\wedge s) \vee \neg(p \wedge r)\\
                &\Leftrightarrow 1
            \end{aligned}
        \]
         
        
        \item $( p \rightarrow  q) \vee  ( p \rightarrow  r) \vee  ( p \rightarrow  s) \rightarrow  ( p \rightarrow  q \vee  r \vee  s )$
        \[
            \begin{aligned}
                &\mathrel{\phantom{=}}(p\rightarrow q)\vee (p \rightarrow r)\vee (p \rightarrow s)\rightarrow (p\rightarrow q\vee r \vee s)\\
                &\Leftrightarrow (\neg p \vee q \vee \neg p \vee r \vee \neg p \vee s )\rightarrow (\neg p \vee q \vee r \vee s)\\
                &\Leftrightarrow (\neg p \vee q \vee r \vee s )\rightarrow (\neg p \vee q \vee r \vee s)\\
                &\Leftrightarrow 1\\
            \end{aligned}
        \]
         
        
        \item  $( p \vee  q \rightarrow  r) \rightarrow  ( p \rightarrow  r) \vee  (q \rightarrow  r)$
        \[
            \begin{aligned}
                &\mathrel{\phantom{=}} (p\vee q \rightarrow r)\rightarrow (p\rightarrow r)\vee(q\rightarrow r)\\
                &\Leftrightarrow \neg(\neg (p\vee q)\vee r)\vee(\neg p \vee r \vee \neg q \vee r)\\
                &\Leftrightarrow ((p\vee q)\wedge \neg r) \vee (\neg p \vee \neg q \vee r)\\
                &\Leftrightarrow (p\vee q \vee \neg p \vee \neg q \vee r)\wedge(\neg r \vee r \vee \neg p \vee \neg q)\\
                &\Leftrightarrow 1\wedge 1\\
                &\Leftrightarrow 1\\
            \end{aligned}
        \]
    \end{enumerate}
    \item 用等值演算证明以下公式是永假式
    \begin{enumerate}
        \item $(q \rightarrow p) \wedge  (\neg p \rightarrow q) \leftrightarrow \neg p$
        \[
            \begin{aligned}
                &\mathrel{\phantom{=}}(q \rightarrow p) \wedge  (\neg p \rightarrow q) \leftrightarrow \neg p\\
                &\Leftrightarrow (\neg q \vee p)\wedge (p \vee q)\leftrightarrow \neg p\\
                &\Leftrightarrow (\neg q \vee q )\wedge p \leftrightarrow \neg p\\
                &\Leftrightarrow 1\wedge p \leftrightarrow \neg p\\
                &\Leftrightarrow p\leftrightarrow \neg p\\
                &\Leftrightarrow 0\\
            \end{aligned}
        \]
 

        \item $ ( p \rightarrow q) \wedge  (q \rightarrow r) \wedge  \neg ( p \rightarrow r)$
        
        
    \end{enumerate}
    \end{enumerate}
\end{document}