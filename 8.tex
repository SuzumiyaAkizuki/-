\documentclass[UTF8]{ctexart}
\usepackage{float}
\usepackage{listings}
\usepackage{graphicx}
\usepackage{xeCJK}
\usepackage{xcolor}
\usepackage{ulem}
\usepackage{CJKfntef}
\usepackage{setspace}
\usepackage{amsmath}
\lstset{
   basicstyle          =   \ttfamily,          % 基本代码风格
   keywordstyle        =   \ttfamily\bfseries,          % 关键字风格
   commentstyle        =   \rmfamily\itshape,  % 注释的风格,斜体
   stringstyle         =   \ttfamily,  % 字符串风格
   flexiblecolumns,                % 别问为什么,加上这个
   numbers             =   left,   % 行号的位置在左边
   showspaces          =   false,  % 是否显示空格,显示了有点乱,所以不现实了
   numberstyle         =   \zihao{-5}	\ttfamily,    % 行号的样式,小五号,tt等宽字体
   showstringspaces    =   false,
   captionpos          =   t,      % 这段代码的名字所呈现的位置,t指的是top上面
   frame               =   lrtb,   % 显示边框
   breaklines      =   true,   % 自动换行,建议不要写太长的行
   columns         =   fixed,  % 如果不加这一句,字间距就不固定,很丑,必须加
}

\title{离散数学第八次作业}
\author{贠启豪\\19375168}
\date{\today}

\begin{document}
    \maketitle
    \begin{enumerate}
        \item 判断以下等值式是否成立,并说明理由。
        \begin{enumerate}
            \item $\forall x(P(x) \leftrightarrow  Q(x)) \Leftrightarrow  \forall xP(x) \leftrightarrow  \forall xQ(x)$
            
            不成立。
            取解释$I_1$:
            \[
                D_I=\{0,1\},P(x)=x,Q(x)=\neg x
            \]
            则$I(\forall x(P(x)\leftrightarrow Q(x)))(v)=0$,但是$I(\forall xP(x)\leftrightarrow \forall xQ(x))(v)=0\leftrightarrow 0=1$

            \item $\forall x(P(x) \rightarrow  Q(x)) \Leftrightarrow  \forall xP(x) \rightarrow  \forall xQ(x)$
            
            不成立。取解释$I_1$,$I(\forall x(P(x)\rightarrow Q(x)))(v)=0$,但是$I(\forall xP(x)\rightarrow \forall xQ(x))(v)=0\rightarrow 0=1$.

            
            \item $\forall xP(x) \Leftrightarrow  P(x)$
            不成立。取解释$I_3$,赋值$v_3$如下:
            \[
                D_{I_2}=\{0,1\},P(x)=x,v(x)=1
            \]
            则$I(\forall xP(X))(v)=0$,但$I(P(x))(v[x/1])=1$.
            \item $\forall x\forall xP(x) \Leftrightarrow  \forall xP(x)$
            
            成立。

            因为$x$不是$\forall xP(x)$的自由变元。对于每个使得$I(\forall xP(x))(v)=1$的解释$I$赋值$v$,有$\forall d\in D_I,I(\forall xP(x))(v[x/d])=1$
            也就是$I(\forall x\forall xP(x))(v)=1$
            
            \item $\forall x(P(x) \leftrightarrow  \forall yQ(y)) \Leftrightarrow  \forall xP(x) \leftrightarrow  \forall yQ(y)$
            
            不成立。对于解释$I_1$,$I(\forall x(P(x)\rightarrow \forall yQ(y)))(v)=I(\forall x(P(x)\leftrightarrow 0))(v)=0$,但是$I(\forall xP(x)\leftrightarrow \forall yQ(y))(v)=0\rightarrow 0=1$
            
            \item $\forall x(P(x) \leftrightarrow  \forall yQ(y)) \Leftrightarrow  \exists xP(x) \leftrightarrow  \forall yQ(y)$
            
            不成立。取解释$I_6$:
            \[
                D_{I_6}=\{0,1\},P(x)=1,Q(x)=1
            \]
            则$I(\forall x(P(x)\leftrightarrow \forall yQ(y)))(v)=0$,但$I(\exists xP(x)\leftrightarrow \forall yQ(y))(v)=1\leftrightarrow1=1$.
        
        
        \end{enumerate}
        
        \item 设 A,B 是任意公式,证明以下等值式。
        \begin{enumerate}
            \item $\exists xA\Leftrightarrow \exists yA^x_y$ ,其中 y 在 A 中不出现。
            \[
                \begin{aligned}
                    &\exists xA\\
                    &\Leftrightarrow \neg(\forall x\neg A)\\
                    &\Leftrightarrow \neg(\forall y\neg A_y^x)\\
                    &\Leftrightarrow \exists yA_y^x\\
                \end{aligned}
            \]
            \item $\exists x(A\rightarrow B)\Leftrightarrow \forall xA\rightarrow \exists xB$
            
            对任意解释I,赋值v:
            \[
                \begin{aligned}
                    &I(\forall xA\rightarrow \exists xB)(v)=0\\
                    &\Leftrightarrow I(\forall xA)(v)=1 \text{ and } I(\exists xB)(v)=0\\
                    &\Leftrightarrow \forall d\in D_I,I(A)(v[x/d])=1 \text{ and } \forall d\in D_I,I(B)(v[x/d])=0\\
                    &\Leftrightarrow \forall d\in D_I,I(A\rightarrow B)(v[x/d_2])=0\\
                    &\Leftrightarrow I(\exists (A\rightarrow B))(v)=0\\
                \end{aligned}    
            \]
            \item $\forall x\forall y(A\vee B)\Leftrightarrow \forall xA\vee \forall yB$,其中 x 不是 B 的自由变元,y 不是 A 的自由变元。
            
            对任意解释I,赋值v:
            \[
                \begin{aligned}
                    &I(\forall xA\vee \forall yB)(v)=0\\
                    &\Leftrightarrow I(\forall xA)(v)=0\text{ and }I(\forall yB)(v)=0\\
                    &\Leftrightarrow \forall d_1\in D_i,I(A)(v[x/d_1])=0\text{ and }\forall b_2\in D_I,I(B)(v[y/b_2])=0\\
                    &\Leftrightarrow \forall d_1\in D_I,\forall d_2\in D_I,I(A\vee B)(v[x/d_1.y/d_2])=0\\
                    &\Leftrightarrow I(\forall x\forall y(A\vee B))(v)=0\\
                \end{aligned}
            \]
            \item $\exists x\exists y(A\wedge B)\Leftrightarrow \exists xA\wedge \exists yB$,其中 x 不是 B 的自由变元,y 不是 A 的自由变元。
            \[
                \begin{aligned}
                    &\exists x\exists y(A\wedge B)\\
                    &\Leftrightarrow \exists x(A\wedge \exists yB)\\
                    &\Leftrightarrow \exists xA \wedge \exists yB\\
                \end{aligned}
            \]
            
            \item $\exists x\forall y(A\rightarrow B)\Leftrightarrow \forall xA\rightarrow \forall yB$,其中 x 不是 B 的自由变元,y 不是 A 的自由变元。
            \[
                \begin{aligned}
                    &\exists x\forall y(A\rightarrow  B)\\
                    &\Leftrightarrow \exists x(A\rightarrow \forall yB)\\
                    &\Leftrightarrow \forall xA\rightarrow \forall yB\\
                \end{aligned}
            \]

            \item $\forall x\forall yA\Leftrightarrow \forall y\forall xA$
            
            这也太显然了吧 真不会证明啊。
        \end{enumerate}
    \end{enumerate}


\end{document}